\section{Постановка задачи}
\subsection{Линейный случай}
Оценить методом Кравчика внешнее множество решений следующей ИСЛАУ
\begin{equation*}
 \begin{cases}
   3x_1+2x_2 = [30, 31]\\
   x_1-[0.8, 1]x_2 = 0\\
 \end{cases}
\end{equation*}
Также необходимо:
\begin{itemize}
    \item определить спектральный радиус матрицы
    \item провести оценку начального бруса решения
\end{itemize}
Провести вычисления и привести иллюстрации:
\begin{itemize}
    \item положения брусов при итерациях
    \item рабочих брусов
    \item расстояния от центров брусов при итерациях до конечной точки алгоритма
\end{itemize}

\subsection{Нелинейный случай}
Оценить методом Кравчика внешнее множество решений следующей ИСЛАУ
\begin{equation*}
 \begin{cases}
   3x_1+2x_2 = [30, 31]\\
   \frac{x_1}{x_2} = [0.8, 1]\\
 \end{cases}
\end{equation*}
Провести вычисления и привести иллюстрации:
\begin{itemize}
    \item положения брусов при итерациях
    \item графики радиусов рабочих брусов
    \item расстояния от центров брусов при итерациях до конечной точки алгоритма
\end{itemize}
Сравнить результаты с линейным случаем.