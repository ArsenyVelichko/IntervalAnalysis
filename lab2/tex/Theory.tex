\section{Теория}
\subsection{Внешнее множество решений}
Внешним множеством решений называется объединенное множество решений, образованное решениями всех точечных систем $F(a,x)=b$
\begin{equation} \label{UniSet}
    \Xi_{\mathrm{uni}}(\textbf{F},\textbf{a}, \textbf{b})=\{x\in\mathbb{R}^n\;|\;(\exists a\in\textbf{a})(\exists b\in\textbf{b})(F(a,x)=b)\}
\end{equation}

\subsection{Метод Кравчика}
Метод Кравчика предназначен для уточнения двухсторонних границ решений систем уравнений, в общем случае нелинейных, заданных на некотором брусе $\textbf{X}\subset \mathbb{IR}$, вида
\begin{equation}
    F(x)=0, \quad \textup{где} \;\; F(x)=\{F_1(x),...,F_n(x)\}^T,\; x=(x_1,...x_n)
\end{equation}
Также данный метод может быть использован для того, чтобы понять, что решений нет.\\
Отображение $\mathcal{K}:\mathbb{ID}\times\mathbb{R}\rightarrow\mathbb{IR}^n$, задаваемое выражением
\begin{equation}
    \mathcal{K}(\textbf{X}, \overline{x}):=\overline{x}-\Lambda*F(\overline{x})-(I-\Lambda*\textbf{L}*(\textbf{X}-\overline{x})
\end{equation}
называеся оператором Кравчика на $\mathbb{ID}$ относительно точки $\overline{x}$.\\
Итерационная схема данного метода выглядит следующим образом
\begin{equation}
    \textbf{X}^{k+1}\leftarrow\textbf{X}^k\cap\mathcal{K}(\textbf{X}^k, \overline{x}^k), \;\; k=0,1,2..., \; x^k\in\textbf{X}^k
\end{equation}
Сходимость данного метода гарантирована при выполнении условия
\begin{equation}
    \rho(I-\Lambda*\textbf{L})<1 - \textup{спектральный радиус меньше единицы}
\end{equation}
Частным случаем данного метода является линейный метод Кравчика, итерационная схема которого выглядит следующим образом:
\begin{equation}
    \textbf{x}^{k+1}=\left(\Lambda*\textbf{b}+(I-\Lambda*\textbf{A})*\textbf{x}^k\right)\cap\textbf{x}^{k}
\end{equation}
$\textbf{A}$ в данном случае является интервальной матрицей коэффициентов соответсвующей ИСЛАУ, а $\textbf{b}$ - вектором свободных членов.\\
В случае линейности системы и выполнения условия $\eta=||I-\Lambda*\textbf{A}||_\infty\leq 1$ в качестве начального приближения можно взять брус 
\begin{equation}
    \textbf{x}^0=([-\theta,\theta],...,[-\theta,\theta])^T, \quad \textup{где}  \;\theta=\frac{||\Lambda \textbf{b}||_\infty}{1-\eta}
\end{equation}