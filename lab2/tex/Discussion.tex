\section{Обсуждение}
\begin{enumerate}
    \item Метод Кравчика для линейной ИСЛАУ показал крайне быструю сходимость. Для достижения конечной точки ему потребовалось всего 2 итерации. Однако при этом достаточно точно достигается лишь верхняя часть интервальной оболочки $\Xi_{\mathrm{uni}}$.
    \item В случае нелинейной интерпретации метод Крачика демонстрирует куда более интересные рузельтаты. Достижение конечной точки занимает порядка 15 итераций. Уточнее ведётся по всем граням одновременно. Финальный брус же отличается от интервальной оболочки $\Xi_{\mathrm{uni}}$ приблизительно на $0.07$ по каждой из граней.
    \item Сравнивая две интерпретации между собой можно сказать, что общая точность, с которой они приблизили $\Xi_{\mathrm{uni}}$ оказалась примерно одинаковой. При этом в достоинства линейного случая можно записать то, что точность достижения одной из граней была достаточна велика, а также что сложность самого метода и поиска начального приближения меньше, чем в нелинейном случае.
\end{enumerate}