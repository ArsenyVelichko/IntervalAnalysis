\section{Обсуждение}
\begin{enumerate}
    \item Информационное множество \eqref{InfSet} представляет из себя шестигранник. Это означает, что  2 из 8 вершин брусов погрешности не оказывают влияния на регрессионную модель.
    \item Все точечные оценки находятся внутри информационного множества и расположены близко друг к другу.
    \item По графику коридора совместных значений \eqref{Corridor} мы можем увидеть те 2 вершины, которые не оказывают влияния. Они обе принадлежат интервалу, соответствующему третьему измерению. Таким образом, третье измерение $x_3=7$ не является граничным и его удаление из выборки не изменит модель. Как и следовало ожидать, коридор начинает сильно расширяться при отдалении от множества измерений.
    \item Обратившись к графику предсказаний \eqref{Predictions} мы видим, что те предсказания, которые находяться вблизи измерений, действительно могли бы неплохо описывать исходную зависимость. В то же время, крайние предсказания имеют большой интервал неопределённости, что говорит нам о том, что экстраполяция исходной зависимости на большой интервал весьма затруднительна.
\end{enumerate}