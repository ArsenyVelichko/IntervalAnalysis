\section{Обсуждение}
\begin{enumerate}
    \item После коррекции правой части расположение  не изменилось. Квадрат $rve$ достаточно хорошо приближает допусковое множество решений по левой границе на оси $x_1$. Квадрат $ive$ почти полностью принадлежит $\Xi_{\mathrm{tol}}$. Подобное расположение оценок и вид допускового множества решения подтсверждают хорошую обусловленность матрицы $(\mathrm{cond}(\mathrm{mid}\,\mathbf{A})=1.6482)$.
    \item Достижение резрешимости рассматриваемой ИСЛАУ путём коррекции матрицы $\mathbf{A}$ возможно только при вырожденности интервала $\mathbf{a}_{22}$. В связи с этим показанное на рисунке \eqref{MatrixCorrSet} допусковое множество вырождается в отрезок прямой. Максимум распознающего функционала при этом смещается.
    \item При коррекции матрицы в целом с увеличением параметра $e$ argmax стремиться к правой вершине треугольника, составленного из центральных точечных уравнений ИСЛАУ. При $e\geq0.25$ задача становится разрешимой. Начиная с этого момента argmax принадлежит нижней грани треугольника. При $e>0.3$ положение argmax перестаёт меняться, несмотря на то, что радиус исходной ИСЛАУ позволяет изменять его до 0.5.
    \item При коррекции первой строки матрицы с увеличением параметра $e$ argmax стремиться к левой нижней вершине треугольника. Начиная с $e=0.25$ задача становится разрешимой.
    \item При коррекции второй строки матрицы с увеличением параметра $e$ argmax стремиться от левой нижней вершины треугольника к своему положению из исходной задачи. Задача является разрешимой только при $e=1$.
    \item При коррекции третьей строки матрицы с увеличением параметра $e$ argmax изменяется в ту же сторону, что и при коррекции матрицы в целом, однако с большей скоростью, а также его <<траектория>> имеет больший угол наклона. Начиная с $e=\frac{1}{3}$ задача становится разрешимой.
\end{enumerate}