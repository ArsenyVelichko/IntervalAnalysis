\section{Обсуждение}
\begin{enumerate}
    \item У функции МакКормика имеется проблема в том, что вблизи минимума она убывает крайне медленно, в связи с чем тот же простейших алгоритм дробления после определённого числа итераций почти не приближает своего значения. В теории эту проблему мог бы решить метод отжига, однако мы видим, что его поведение при выбранных нами параметрах аналогично простейшему алгоритму дробления, и он также начинает <<рыскать>> по дну оврага вместо монотонного стремления к минимуму.\\
    На графике расстояния до минимума также видна одна из проблем метода отжига - он может остановить своё выполнение не в наилучшей точке рабочего списка. Причиной этого служит вероятность \eqref{Probability}, из-за которой мы всегда имеем шанс сделать ведущим брус, чья оценка больше текущей.
    \item Функция Изома является достаточно простой для оптимизации. Об этом свидетельствует и быстрая сходимость обоих методов. При этом мы видим, что простейший алгоритм дробления за 100 итераций успевает достигнуть точности порядка $10^{-8}$, в то время как метод отжига лишь $10^{-4}$. Это говорит нам о том, что метод отжига на тривиальных функциях является далеко не самым предпочтительным кандидатом.
    \item Самой сложной из тестируемых функций является функция «шестигорбый верблюд». Она имеет 6 локальных минимумов, 2 из которых - глобальные. Глобальные минимумы мы наблюдаем ближе к центру, они имеют координаты $(0.08984, -0.71266)$ и $(-0.08984, 0.71266)$. Ключевая трудность данной функции заключается в том, что в первую очередь находятся локальные минимумы, а для перехода к глобальным тому же простейшему алгоритму дробления требуется порядка 4000 итераций. Мы видим, что оба метода проделали порядка 5000 итераций и приблизились на примерно одно расстояние к минимуму. Можно акцентировать внимание на том, что метод отжига достиг положения, относительно близкого к глобальному минимуму, ещё на отметке в 2000 итераций, однако после всё равно предпочёл перейти к более дальнему брусу. Опять же в теории подобного поведения можно избежать более точным подбором параметров.
    \item Подводя итог по трём исследованным функциям стоит сказать, что применение метода отжига имеет наибольшее обоснование в случае сложных функций с несколькими локальными минимумами. Данный метод в процессе написания данной работы проявил себя крайне нестабильным образом и подбор описанных выше параметров проводился в первую очерель эмпирическим путём, однако в конечном итоге на функции МакКормика и функции <<шестигорбый верблюд>> он показал результаты, соизмеримые простейшему алгоритму дробления.
\end{enumerate}