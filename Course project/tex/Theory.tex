\section{Теория}
\subsection{Естественное интервальное расширение}
Интервальное расширение элементарного функционального выражения, которое получается в результате замены его аргументов на интервалы их изменения, а арифметических операций и элементарных функций на их интервальные аналоги и расширения называется \textit{естественным интервальным расширением}.
\subsection{Оценка расстояния между функцией и её естественным интервальным расширением}
Пусть $f:\mathbb{R}^n\rightarrow\mathbb{R}$ является липшицевым по форме на $\mathbf{X}\in\mathbb{IR}^n$.\\
Пусть $\textbf{\textit{f}}_{\mathrm{nat}}$ - естественное интервальное расширение $f$.\\
Пусть $f(\mathbf{x})=\{f(x)\;|\;x\in\mathbf{x}\}$, тогда имеет место оценка
\begin{equation} \label{Estimate}
    \mathrm{dist}(\textbf{\textit{f}}_{\mathrm{nat}},f(\mathbf{x}))\leq C||\mathrm{wid}(\mathbf{x})||
\end{equation}
для любого бруса $\mathbf{x}\subset\mathbf{X}$ и некоторой константы $C$.
\subsection{Задача глобальной отпимизации}
Пусть $f:\mathbf{X}\rightarrow\mathbb{R}$, где $\mathbf{X}\subseteq\mathbb{IR}^n$ - прямоугольный брус с гранями параллельными осям.\\
Требуется найти величину $f^*\in\mathbb{R}$, такую что
\begin{equation}
    ||f^*-\inf_{x\in\mathbf{X}}f(x)||<\varepsilon
\end{equation}
для любого заранее заданного $\varepsilon>0$.
\subsection{Интервальный метод отжига}
Алгоритм схема интервального метода обжига:
\begin{table}[H]
\centering
\begin{tabular}{|l| }
 \hline
\multicolumn{1}{|c|}{Вход}\\
     Брус $\mathbf{X}\in\mathbb{IR}^n$. Начальное $T_0$ и конечное $T_{\mathrm{fin}}$ значения <<температуры>>.\\
     Интервальное расширение $\textbf{\textit{F}}:X\rightarrow\mathbb{IR}^n$.\\
 \hline
 \multicolumn{1}{|c|}{Выход}\\
 Оценка $F^*$ глобального минимума функции $F$ на $\mathbf{X}$.\\
 \hline
 \multicolumn{1}{|c|}{Алгоритм}\\
 Присваиваем $\mathbf{Y} \leftarrow \mathbf{X}$ и $T\leftarrow T_0$;\\
 Назначаем целочисленную величину $N_T$ - \\<<кол-во испытаний на один температурный уровень>>;\\
 Вычисляем $\textbf{\textit{F}}(\mathbf{Y})$ и инициализируем список $\mathcal{L}$ записью $\{(\mathbf{Y},\underline{\textbf{\textit{F}}(\mathbf{Y})})\}$;\\
$\mathbf{DO\;WHILE}$ $(T>T_{\mathrm{fin}})$\\
\qquad $\mathbf{DO\;FOR}$ $j=1$ $\mathbf{TO}$ $N_T$\\
\qquad \qquad Случайно выбираем из $\mathcal{L}$ запись $\{(\mathbf{Z},\underline{\textbf{\textit{F}}(\mathbf{Z})})\}$ по правилу $\mathcal{S}(\mathbf{Y})$\\
\qquad\qquad $\mathbf{DO}$ (с вероятностью $P_T(\mathbf{Y}, \mathbf{Z})$)\\
\qquad\qquad\qquad Рассекаем $\mathbf{Z}$ по самой длинной компоненте пополам\\
\qquad\qquad\qquad на брусы-потомки $\mathbf{Z}'$ и $\mathbf{Z}''$;\\
\qquad\qquad\qquad Вычисляем $\textbf{\textit{F}}(\mathbf{Z'})$ и $\textbf{\textit{F}}(\mathbf{Z''})$;\\
\qquad\qquad\qquad Удаляем запись $\{(\mathbf{Z},\underline{\textbf{\textit{F}}(\mathbf{Z})})\}$ из списка $\mathcal{L}$;\\
\qquad\qquad\qquad Помещаем записи $\{(\mathbf{Z},\underline{\textbf{\textit{F}}(\mathbf{Z'})})\}$ и $\{(\mathbf{Z},\underline{\textbf{\textit{F}}(\mathbf{Z''})})\}$ в список $\mathcal{L}$;\\
\qquad\qquad\qquad Обозначаем через $\{(\mathbf{Y},\underline{\textbf{\textit{F}}(\mathbf{Y})})\}$ ту из записей $\{(\mathbf{Z},\underline{\textbf{\textit{F}}(\mathbf{Z'})})\}$ \\
\qquad\qquad\qquad и $\{(\mathbf{Z},\underline{\textbf{\textit{F}}(\mathbf{Z''})})\}$, которая имеет меньшее значение\\
\qquad\qquad\qquad второго поля;\\
\qquad\qquad $\mathbf{END\;DO}$\\
\qquad $\mathbf{END\;DO}$\\
\qquad Уменьшаем значение температуры $T\leftarrow\alpha T$;\\
$\mathbf{END\;DO}$\\
$F^*\leftarrow\underline{\textbf{\textit{F}}(\mathbf{Y})})$;\\\\
\hline
\end{tabular}
\caption{Алгоритм интервального метода отжига}
\label{SimAnnAlgo}
\end{table}
Вероятность дробления выбранного бруса $\mathbf{Z}$ задаётся формулой
\begin{equation} \label{Probability}
    P_T(\mathbf{Y},\mathbf{Z})=
    \begin{cases}
    1, & \text{если} \;\Delta\textbf{\textit{F}}< 0,\\
   \exp(-\frac{\textbf{\textit{F}}}{kT}), & \text{если}\; \Delta\textbf{\textit{F}}\geq 0
    \end{cases}
\end{equation}
где $\Delta\textbf{\textit{F}}=\textbf{\textit{F}}(\mathbf{Z})-\textbf{\textit{F}}(\mathbf{Y})$ - приращение оценки оптимума, обеспечиваемое новым брусом приближения.\\
Правило $\mathcal{S}(\mathbf{Y})$ в алгоритме \eqref{SimAnnAlgo} - это случайный выбор бруса $\mathbf{Z}$ из рабочего списка. Оно зависит от ведущего бруса $\mathbf{Y}$ (а также ведущей оценки) и может быть организовано самым различным образом в зависимости от наличия априорной информации о целевой функции.
\newpage