\section{Постановка задачи}
\subsection{Выяснение радиуса элементов матрицы, при котором она становится особенной}
\subsubsection{Линейная и полиномиальная регрессия}
Задача регрессии может быть записана в следующем виде
\begin{equation}
    \textbf{y}=\textbf{X}\beta
\end{equation}
Пусть $\textbf{X}$ - интервальная матрица и
\begin{equation}
\mathrm{mid}(\textbf{X})=
\begin{pmatrix}
1 & 1 \\
1.1 & 1 
\end{pmatrix}
\end{equation}
Необходимо рассмотреть матрицу вида
\begin{equation}
\textbf{X}=
\begin{pmatrix}
[1-\varepsilon, 1+\varepsilon] & 1 \\
[1.1-\varepsilon, 1.1+\varepsilon] & 1 
\end{pmatrix}
\end{equation}
и определить при каком радиусе она содержит особенную матрицу.\\\\
\subsubsection{Задачи томограции}
При решении задач томографии, имеем уравнения типа
\begin{equation}
    \textbf{A}x=\textbf{b}
\end{equation}
Необходимо рассмотреть интервальную матрицу $2\times2$
\begin{equation}
\textbf{A}=
\begin{pmatrix}
[1-\varepsilon, 1+\varepsilon] & [1-\varepsilon, 1+\varepsilon] \\
[1.1-\varepsilon, 1.1+\varepsilon] & [1.1-\varepsilon, 1.1+\varepsilon] 
\end{pmatrix}
\end{equation}
и определить при каком радиусе она содержит особенную матрицу.
\subsection{Глобальная оптимизация}
При помощи простейшего метода глобальной оптимизации найти точки глобального минимума для функции МакКормика 
\begin{equation}\label{McCormick}
\begin{split}
    f(x,y)=\sin(x+y)+(x-y)^2-1.5x+2.5y+1, 
    \quad    &-1.5\leq x \leq 4\\
        &-3\leq y \leq 4
\end{split}
\end{equation}
\\
И функции Химмельблау
\begin{equation}\label{Himmelblau}
    f(x, y) = (x^2 + y - 11)^2 + (x + y^2 - 7)^2, \quad -5 \leq x,y \leq 5
\end{equation}
\\
Также необходимо привести иллюстрации:
\begin{itemize}
    \item положения брусов из рабочего списка алгоритма и положения их центров
    \item графики радиусов рабочих брусов в логарифмическом масштабе
    \item расстояния до точки минимума в логарифмических координатах
\end{itemize}
