\section{Теория}
\subsection{Определения}
\begin{itemize}
    \item Середина матрицы $\mathrm{mid}(\textbf{A})=\{A\;|\;a_{ij}=\mathrm{mid}(\textbf{a}_{ij})\}$
    \item Радиус матрицы $\mathrm{rad}(\textbf{A})=\{A\;|\;a_{ij}=\mathrm{rad}(\textbf{a}_{ij})\}$
    \item Матрица $\textbf{A}\in \mathbb{IR}$ называется особенной, если $\exists A \in \textbf{A} : det(A)=0$.
    \item Числа $\sigma_1...\sigma_k$, равные квадратным корням из собственных значений матрицы $AA^T$, называется cингулярными числами матрицы $A$.
    \item Множество вершин интревальной матрицы\\ $\mathrm{vert}(\textbf{A})=\{A\in\mathbb{IR}^{m\times n} \;|\; A=(a_{ij}) \, a_{ij}\in\{\underline{\textbf{a}}_{ij}, \overline{\textbf{a}}_{ij}\}\}$
\end{itemize}
\subsection{Критерий Баумана}
Интервальная матрица $\textbf{A}$ неособенна тогда и только тогда, когда 
\begin{equation}
    (\mathrm{det}(A'))*(\mathrm{det}(A''))>0 \quad \forall A',A''\in \mathrm{vert}(A)
\end{equation}
\subsection{Признак Румпа}
Если для интервальной матрицы $\textbf{A}\in \mathbb{IR}^{m\times n}$ имеет место
\begin{equation}
    \sigma_{\mathrm{max}}(\mathrm{rad}(\textbf{A}))<\sigma_{\mathrm{min}}(\mathrm{mid}(\textbf{A}))
\end{equation}
Тогда $\textbf{A}$ неособенна.


