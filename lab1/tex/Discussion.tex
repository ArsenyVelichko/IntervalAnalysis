\section{Обсуждение}
\begin{enumerate}
    \item В обеих задачах значения критерия Баумана и признака Румпа согласуются между собой. В задаче регрессии признак Румпа даёт более узкий интервал для $\varepsilon$, а в задаче томографии - почти идентичный, что и критерий Баумана.
    \item \label{important}На графике радиусов брусов для функции МакКормика мы видим, что монотонное убывание отсутствует. Это означает, что мы не всегда дробим брусы <<вглубь>>, и периодически выбираем новый ведущий брус, не являющийся потомком предыдущего. Данное обстоятельство делает график расстояния до минимума куда менее релевантным, однако мы всё равно можем увидеть его убывание до порядка $10^{-1}$.
    \item На функции МакКормика скорость сходимости алгоритма сильно падает, в связи с чем увеличение кол-ва итераций больше 200 имеет малый смысл. Причиной этого является пункт обсуждения \eqref{important} и широкое <<дно>> в овражной структуре данной функции, внутри которого она крайне мало изменяет своё значение.
    \item Для функции МакКормика алгоритм остановился на брусе $\begin{pmatrix}
    [   -0.3469,   -0.2812] \\
    [   -1.3750,   -1.2874] \end{pmatrix}$, минимальное значение в котором равно $-2.25$, в то время как реальный минимум  $f(-0.54719, -1.54719)=-1.9133$. Тот факт, что брус не содержит точку минимума вновь отсылает нас к пункту обсуждения \eqref{important}, и означает лишь то, что мы в дальнейшем бы перешли к брусу, который бы содержал точку минимума.
    \item Для функции Химмельблау радиусы брусов монотонно убывают, это означает, что для каждого из минимумов мы постоянно выбираем в качестве нового ведущего бруса потомка предыдущего.
    \item Для функции Химмельблау алгоритм сошёлся к вырожденному брусу $[ 3.5844,   -1.8481]$, который описывает соответствующий минимум с точностью порядка $10^{-5}$.
    \item Можно сделать вывод, что работа алгоритма сильно зависит от поведения функции в окрестности минимума, в связи с чем результат может сильно уступать в точности классическим численным методам. Безусловным плюсом данного алгоритма является возможность вычисления сразу нескольких локальных экстремумов.
\end{enumerate}