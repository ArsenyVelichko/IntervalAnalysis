\section{Обсуждение}
\begin{enumerate}
    \item Решение полученное по теореме Зюзина достаточно точно описывает допусковое множество решений - радиус конечного бруса равен минимальной диагонали между его вершинами. По графику радиусов брусов \eqref{ZuzRad} мы видим, что сходимость к конечному брусу происходит с двух сторон, одна итерация полностью лежит в $\Xi_{\mathrm{tol}}$, а последующая выходит за его границы. Сходимость данного процесса занимает порядка 10 итераций.
    \item При решении задачи \eqref{System1} субдифференциальный метод Ньютона хорошо описывает $\Xi_{\mathrm{tol}}$ в той его части, где $x_1>0$. При этом конечный брус достигается всего за 3 итерации.
    \item При решении задачи \eqref{System2} субдифференциальный метод Ньютона зацикливается между двумя брусьями, которые обозначены на рисунке \eqref{Cycle} синим и красным цветом. Оба они выходят за границы допускового множества. Однако границы $\Xi_{\mathrm{tol}}$ лучше приближает синий брус, так как две его нижние вершины лежат на гранях данного множества. Для достижения цикла требуется порядка 10 итераций.
    \item Сравнивая результаты для задач \eqref{System1} и \eqref{System2} можно сказать, что даже малые изменения правой части могут сильно повлиять на результаты  субдифференциального метода Ньютона, что не лучшим образом характеризует его устойчивость.
\end{enumerate}