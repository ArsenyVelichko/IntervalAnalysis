\section{Теория}
\subsection{Теорема Зюзина}
Пусть в интервальной линейной системе уравнений 
\begin{equation}
\mathbf{C}x = \mathbf{d}    
\end{equation}
с $\mathbf{C}\in \mathbb{KR}^{n\times n}$ и $\mathbf{d}\in\mathbb{KR}^{n} $ правильная проекция матрицы $\mathbf{C}$ имеет диагональное преобладание. Тогда формальное решение существует и единственно.\\
\\
\textbf{Итерационная схема}\\
Пусть\\
$\mathbf{D} -$ диагональная матрица $\mathrm{diag}\{\mathbf{c_{11}},\mathbf{c_{22}},...,\mathbf{c}_{nn}\}$,\\
$\mathbf{E} -$ матрица, полученная из $\mathbf{C}$ занулением её диагональных элементов.\\
Таким образом, $\mathbf{C}=\mathbf{D}+\mathbf{E}$, а формальные решения исходной системы очевидно совпадают с формальными решениями системы
\begin{equation}
    \mathbf{D}x+\mathbf{E}x=\mathbf{d}
\end{equation}
которая, в свою очередь равносильна
\begin{equation}
    \mathbf{D}x=\mathbf{d}\ominus\mathbf{E}x
\end{equation}
Взяв далее какой-нибудь вектор $\mathbf{x}^{(0)}$, мы можем организовать итерационный процесс
\begin{equation}
    \mathbf{x}^{(k+1)}\leftarrow\left(\mathrm{inv}\;\mathbf{D}\right)\left(\mathbf{d}\ominus\mathbf{E}\mathbf{x}^{(k)}\right), \quad k=0,1,2,...
\end{equation}
c $\left(\mathrm{inv}\;\mathbf{D}\right):=\mathrm{diag}\{\mathrm{inv}\;\mathbf{c_{11}},\mathrm{inv}\;\mathbf{c_{22}},...,\mathrm{inv}\;\mathbf{c}_{nn}\}$ и по теореме Шрёдера о неподвижной точке, он будет сходиться к единственной неподвижной точке отображения
\begin{equation}
    \mathbf{x}\mapsto\left(\mathrm{inv}\;\mathbf{D}\right)\left(\mathbf{d}\ominus\mathbf{E}\mathbf{x}\right)
\end{equation}
в силу диагонального преобладания в $\mathbf{C}$.
\subsection{Субдифференциальный метод Ньютона}
Пусть
\begin{equation}
    \mathcal{F}(y)=\mathrm{sti}\:(\mathbf{C}\;\mathrm{sti}^{-1}\:(y))-y+\mathrm{sti}\:(\mathbf{d)}
\end{equation}
Тогда итерационная схема метода выглядит следующим образом
\begin{equation}
    \mathbf{x}^{(k+1)}\leftarrow x^{(k)}-\tau\left(D^{(k)}\right)^{-1}\mathcal{F}(x^{(k)})
\end{equation}
где $\tau\in[0,1]$, а $D^{(k)}$ - некоторый субградиент отображения $\mathcal{F}$ в точке $x^{(k)}$.