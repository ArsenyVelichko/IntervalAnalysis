\section{Постановка задачи}
\subsection{Получение решения по теореме Зюзина}
Построить итерационную схему с разложением на матрицы диагональную и недиагональную части для следующей ИСЛАУ
\begin{equation*}
 \begin{cases}
   [2,4]x_1+[-2,1]x_2 = [-2, 2]\\
   [-2, 1]x_1+[2, 4]x_2 = [-2. 2]\\
 \end{cases}
\end{equation*}
Провести вычисления и привести иллюстрации:
\begin{itemize}
    \item брусов итерационного процесса
    \item радиусов решения в зависимости от номера итерации
\end{itemize}
\subsection{Получение решения субдифференциальным методом Ньютона}
Для ИСЛАУ
\begin{equation} \label{System1}
 \begin{cases}
   [3,4]x_1+[5,6]x_2 = [-3, 3]\\
   [-1, 1]x_1+[-3, 1]x_2 = [-1. 2]\\
 \end{cases}
\end{equation}
построить итерационную схему субдифференциального метода Ньютона.\\
Провести вычисления и привести иллюстрации:
\begin{itemize}
    \item брусов итерационного процесса
\end{itemize}
Аналогично поступить с ИСЛАУ
\begin{equation} \label{System2}
 \begin{cases}
   [3,4]x_1+[5,6]x_2 = [-3, 4]\\
   [-1, 1]x_1+[-3, 1]x_2 = [-1. 2]\\
 \end{cases}
\end{equation}
Сравнить результаты для двух ИСЛАУ.